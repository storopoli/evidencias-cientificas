%--- Preamble ---------------------------------------------------------%
% Load LaTeX packages
\documentclass[10pt, xcolor=dvipsnames]{beamer} 
\usepackage[absolute, overlay]{textpos}                           % supports floating text in any location
\usepackage{tikz}
\usetikzlibrary{graphs}
\usetikzlibrary{graphdrawing}
\usegdlibrary{force}

% Customize theme attributes
\useoutertheme{infolines}					                     % adds 3 box footer
\usetheme[height=7mm]{Rochester} 				                 % choose theme
\setbeamertemplate{blocks}[rounded][shadow=true] 		         % rounded theorem box with shadow
\setbeamertemplate{caption}[numbered]                             % enable counting of tables/figures

% Define template colors
\definecolor{QPblue}{RGB}{0,25,100}                               % define QP blue using RGB code
\definecolor{QPgreen}{RGB}{0,153,110}                             % define QP green using RGB code
\setbeamercolor{title}{fg=white, bg=QPblue}                       % define title page box color
\setbeamercolor{frametitle}{fg=white, bg=QPblue}                  % define frame title color
\setbeamercolor{normal text}{fg=black}                            % define standard font color
\setbeamercolor{author in head/foot}{fg=QPblue, bg=QPblue!75}    % define infoline 1st box color
\setbeamercolor{title in head/foot}{fg=QPblue, bg=QPblue!60}     % define info line 2nd box color
\setbeamercolor{date in head/foot}{fg=QPblue, bg=QPblue!30}	  % define infoline 3rd box color
\setbeamercolor{block title}{fg=QPblue!50!QPgreen, bg=QPblue!30} % define theorem box title color
\setbeamercolor{block body}{fg=QPblue!50!QPgreen, bg=gray!10}	 % define theorem box body color
\setbeamercolor{local structure}{fg=QPblue!75}		           % define bullet and enumerate list colors

% Define global environments
\newenvironment{reference}[2]{                                    % define environment for footnotes
  \begin{textblock*}{\textwidth}(#1, #2)
      \tiny\it\bgroup\color{red!70!QPblue}}{\egroup\end{textblock*}}

% Define title page logo and project metadata
\titlegraphic{\includegraphics[width=3cm]{UNINOVE_LOGO.JPG}\hspace*{0cm}~   
}

\title{Beamer - UNINOVE} 
\subtitle{Creating A Custom Presentation Template with \LaTeX{} Code}
\author{Jose Storopoli} 
\institute[Cidades Inteligentes e Sustentáveis]{
   \textcolor{QPblue!75}{Universidade Nove de Julho \\
   UNINOVE \\
   São Paulo \\ 
   Brasil \\ [1ex]
   \texttt{josees@uni9.pro.br}}
} 
\date{Setembro 2020}

\begin{document}

%--- Title Page -------------------------------------------------------%
 
\begin{frame}[plain]
  \titlepage
\end{frame}

%--- Slide 1 ----------------------------------------------------------%
 
\begin{frame}{Bullets}
 
Here is a sample slide that shows what \emph{itemized} and \emph{enumerated} lists look:  
 
\begin{columns}
  \begin{column}{0.45\textwidth}
  \begin{itemize}
    \item itemized item 1
    \item itemized item 2
    \item itemized item 3
  \end{itemize}
  \end{column}
 
  \begin{column}{0.45\textwidth}
  \begin{enumerate}
    \item enumerated item 1
    \item enumerated item 2
    \item enumerated item 3
  \end{enumerate}
  \end{column}
\end{columns}
~\\[2ex]
 
The sample code also defines 2 columns.
 
\end{frame}

 %--- Slide 2 -------------------------------------------------------------%
 
\begin{frame}{Floating Text} 
The \LaTeX{} package \texttt{textpos} makes it possible to put text objects in arbitrarily prescribed places.\\[2ex]
 
With floating text, you define the environment name, font size, type, and color.\\[2ex]
 
In this example, The \texttt{reference} environment is created to take two input arguments, which specify \texttt{x} and \texttt{y} text position.\\[2ex]  
 
Slide coordinates are defined relative to the top left corner.  A Beamer slide has dimensions 128mm by 98mm.  Trial and error ensures the ad-hoc text lands where you want.\\[2ex]
 
Example below:
 
\begin{reference}{4mm}{75mm}
      V. Jikov, S. Kozlov and O. Olenik, Homogenization of differential operators and integral functionals, Springer, 1994.
\end{reference} 
\end{frame}
 
 
  %--- End -------------------------------------------------------------%
\end{document}
