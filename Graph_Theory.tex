%--- Preamble ---------------------------------------------------------%
% Load LaTeX packages
\documentclass[10pt, xcolor=dvipsnames]{beamer} 
\usepackage[absolute, overlay]{textpos}                           % supports floating text in any location
\usepackage{tikz}
\usepackage{tkz-berge}						       % graph drawing macro
\usetikzlibrary{graphs}
\usetikzlibrary{graphdrawing}
\usetikzlibrary{graphs.standard}
\usetikzlibrary{quotes}
\usegdlibrary{force}


% Customize theme attributes
\useoutertheme{infolines}					                     % adds 3 box footer
\usetheme[height=7mm]{Rochester} 				                 % choose theme
\setbeamertemplate{blocks}[rounded][shadow=true] 		         % rounded theorem box with shadow
\setbeamertemplate{caption}[numbered]                             % enable counting of tables/figures

% Define template colors
\definecolor{QPblue}{RGB}{0,25,100}                               % define QP blue using RGB code
\definecolor{QPgreen}{RGB}{0,153,110}                             % define QP green using RGB code
\setbeamercolor{title}{fg=white, bg=QPblue}                       % define title page box color
\setbeamercolor{frametitle}{fg=white, bg=QPblue}                  % define frame title color
\setbeamercolor{normal text}{fg=black}                            % define standard font color
\setbeamercolor{author in head/foot}{fg=QPblue, bg=QPblue!75}    % define infoline 1st box color
\setbeamercolor{title in head/foot}{fg=QPblue, bg=QPblue!60}     % define info line 2nd box color
\setbeamercolor{date in head/foot}{fg=QPblue, bg=QPblue!30}	  % define infoline 3rd box color
\setbeamercolor{block title}{fg=QPblue!50!QPgreen, bg=QPblue!30} % define theorem box title color
\setbeamercolor{block body}{fg=QPblue!50!QPgreen, bg=gray!10}	 % define theorem box body color
\setbeamercolor{local structure}{fg=QPblue!75}		           % define bullet and enumerate list colors

% Define global environments
\newenvironment{reference}[2]{                                    % define environment for footnotes
  \begin{textblock*}{\textwidth}(#1, #2)
      \tiny\it\bgroup\color{red!70!QPblue}}{\egroup\end{textblock*}}

% Define title page logo and project metadata
\titlegraphic{\includegraphics[width=3cm]{UNINOVE_LOGO.JPG}\hspace*{0cm}~   
}

\title{Teoria dos Grafos - UNINOVE} 
\subtitle{Creating A Custom Presentation Template with \LaTeX{} Code}
\author{Jose Storopoli} 
\institute[Cidades Inteligentes e Sustentáveis]{
   \textcolor{QPblue!75}{Universidade Nove de Julho \\
   UNINOVE \\
   São Paulo \\ 
   Brasil \\ [1ex]
   \texttt{josees@uni9.pro.br}}
} 
\date{Setembro 2020}

\begin{document}

%--- Title Page -------------------------------------------------------%
 
\begin{frame}[plain]
  \titlepage
\end{frame}

%--- Slide 1 ----------------------------------------------------------%
\begin{frame}
\frametitle{Graphs using TikZ}

\begin{figure}

\tikz \graph [spring layout, nodes={draw,circle}] {
a[as=$a_1$, red] ->[thick, red, "foo"] b -- c -- a; %node+edge style + label
d -- e;
f -- g -- h -- d -- f;
e -- g; };

\caption{\label{fig:graph_8}Graph with 8 vertices.}
\end{figure}
\end{frame}

%--- Slide 2 ----------------------------------------------------------%
\begin{frame}
\frametitle{Graph Macros using TikZ}
A complete graph

\begin{figure}

\tikz \graph {subgraph K_n [n=6, clockwise, nodes={draw,circle}] };
\caption{\label{fig:graph_k_6}Complete Graph with 6 vertices.}

\end{figure}
\end{frame}

%--- Slide 3 ----------------------------------------------------------%
\begin{frame}
\frametitle{Graphs using tkz-berge}
A complete graph

\begin{figure}

\begin{tikzpicture}[scale=0.4]
  \grComplete[Math]{7}
  
\end{tikzpicture}

\caption{\label{fig:complete_7}Complete Graph with 7 vertices.}
\end{figure}

\end{frame}

%--- Slide 4 ----------------------------------------------------------%
\begin{frame}
\frametitle{Graphs using tkz-berge}
A star graph

\begin{figure}

\begin{tikzpicture}[scale=0.4]

%\grStar[prefix=s, Math = true]{4}
\SetVertexNoLabel        % remove labels from Nodes
\grStar[]{4}
\AssignVertexLabel[Math=true]{a}{s_2, s_3, s_4, s_1}
  
\end{tikzpicture}

\caption{\label{fig:star_4}Star Graph with 4 vertices.}
\end{figure}

\end{frame}



 %--- End -------------------------------------------------------------%

\end{document}
